\documentclass[11pt]{article}
\usepackage{amssymb, amsthm, amsmath}
\usepackage{bm}
\usepackage{graphicx}
\usepackage[authoryear]{natbib}
\usepackage{bm}
\usepackage{verbatim}
\usepackage{lineno}
\usepackage{times}
\usepackage{soul}
\usepackage{color}

\usepackage[left=1in,top=1in,right=1in]{geometry}
\newcommand{\btheta}{ \mbox{\boldmath $\theta$}}
\newcommand{\bmu}{ \mbox{\boldmath $\mu$}}
\newcommand{\balpha}{ \mbox{\boldmath $\alpha$}}
\newcommand{\bbeta}{ \mbox{\boldmath $\beta$}}
\newcommand{\bdelta}{ \mbox{\boldmath $\delta$}}
\newcommand{\blambda}{ \mbox{\boldmath $\lambda$}}
\newcommand{\bgamma}{ \mbox{\boldmath $\gamma$}}
\newcommand{\brho}{ \mbox{\boldmath $\rho$}}
\newcommand{\bpsi}{ \mbox{\boldmath $\psi$}}
\newcommand{\bepsilon}{ \mbox{\boldmath $\epsilon$}}
\newcommand{\bomega}{ \mbox{\boldmath $\omega$}}
\newcommand{\bOmega}{ \mbox{\boldmath $\Omega$}}
\newcommand{\bDelta}{ \mbox{\boldmath $\Delta$}}
\newcommand{\bSigma}{ \mbox{\boldmath $\Sigma$}}
\newcommand{\bPsi}{\mbox{\boldmath $\Psi$}}
\newcommand{\bOne}{\mbox{\boldmath $1$}}
\newcommand{\omu}{\overline{\mu}}
\newcommand{\oSigma}{\overline{\Sigma}}
\newcommand{\Yt}{{\tilde Y}}
\newcommand{\bA}{ \mbox{\bf A}}
\newcommand{\bP}{ \mbox{\bf P}}
\newcommand{\bx}{ \mbox{\bf x}}
\newcommand{\bX}{ \mbox{\bf X}}
\newcommand{\bB}{ \mbox{\bf B}}
\newcommand{\bZ}{ \mbox{\bf Z}}
\newcommand{\by}{ \mbox{\bf y}}
\newcommand{\bY}{ \mbox{\bf Y}}
\newcommand{\bz}{ \mbox{\bf z}}
\newcommand{\bh}{ \mbox{\bf h}}
\newcommand{\br}{ \mbox{\bf r}}
\newcommand{\bt}{ \mbox{\bf t}}
\newcommand{\bs}{ \mbox{\bf s}}
\newcommand{\bb}{ \mbox{\bf b}}
\newcommand{\bL}{ \mbox{\bf L}}
\newcommand{\bu}{ \mbox{\bf u}}
\newcommand{\bv}{ \mbox{\bf v}}
\newcommand{\bV}{ \mbox{\bf V}}
\newcommand{\bW}{ \mbox{\bf W}}
\newcommand{\bG}{ \mbox{\bf G}}
\newcommand{\bH}{ \mbox{\bf H}}
\newcommand{\bw}{ \mbox{\bf w}}
\newcommand{\bo}{ \mbox{\bf o}}
\newcommand{\bfe}{ \mbox{\bf e}}
\newcommand{\iid}{\stackrel{iid}{\sim}}
\newcommand{\indep}{\stackrel{indep}{\sim}}
\newcommand{\calR}{{\cal R}}
\newcommand{\calG}{{\cal G}}
\newcommand{\calD}{{\cal D}}
\newcommand{\calS}{{\cal S}}
\newcommand{\calB}{{\cal B}}
\newcommand{\calA}{{\cal A}}
\newcommand{\calT}{{\cal T}}
\newcommand{\calO}{{\cal O}}
\newcommand{\argmax}{{\mathop{\rm arg\, max}}}
\newcommand{\argmin}{{\mathop{\rm arg\, min}}}
\newcommand{\Frechet}{\mbox{Fr$\acute{\mbox{e}}$chet }}
\newcommand{\Matern}{\mbox{Mat$\acute{\mbox{e}}$rn }}
\newcommand{\ballunion}{B_a(\bs_1) \cup B_b(\bs_2) }

\newcommand{\beq}{ \begin{equation}}
\newcommand{\eeq}{ \end{equation}}
\newcommand{\beqn}{ \begin{eqnarray}}
\newcommand{\eeqn}{ \end{eqnarray}}

\pdfpageheight 11in
\pdfpagewidth 8.5in
\linespread{1.0}

\begin{document}\linenumbers
\citep{Gneiting2011}

p. 415 (Table 4) gives suggested threshold weighting for Brier and Quantile scores

\section{Continuous Ranked Probability Scores (CRPS)}
\begin{itemize}
    \item For each of the CRPS versions, they still integrate over all $q$ and $\tau$
    \item For these to be strictly proper, we need a finite first moment ($\xi < 1$)
\end{itemize}


\subsection{Brier scores}
The Brier probability scores is given by 
\begin{align}
    \text{PS}[ F(q), I(y \le q) ] = [ F(q) - I(y \le q) ]^2
\end{align}
where $q$ is a quantile of interest.

The CRPS with Brier score is
\begin{align}
    S(f, y) = \int_{-\infty}^{\infty} \text{PS}[F(q), I(y \le q) ] u(q) \text{d}q
\end{align}
where $u(q)$ is a weighting function (see \ref{sec:weighting}).

Discretized (integral w.r.t. a discrete Stieltjes measure)
\begin{align}
    S(f, y) = \frac{ q_u - q_\ell }{ I - 1 } \sum_{i = 1}^I u(q_i) \text{PS}[ F(q), I(y \le q) ] 
\end{align}
where $(q_\ell, q_u)$ is the range of interest

\subsection{Quantile scores}
The quantile score is given by
\begin{align}
    \text{QS}_\tau [ q(\tau), y ] = 2 \{ I[y \le q(\tau) ] \} [q(\tau) - y]
\end{align}
where $q(\tau)$ is the $\tau$th quantile of the density.

The CRPS with quantile score is
\begin{align}
    S(f, y) = \int_{0}^{1} \text{QS}_\tau (q(\tau), y) v(\tau) \text{d}\tau
\end{align}
where $v(\tau)$ is a weighting function (see \ref{sec:weighting}).

Discretized (integral w.r.t. a discrete Stieltjes measure)
\begin{align}
    S(f, y) = \frac{ 1 }{ J - 1} \sum_{ j = 1 }^{ J-1 } v(\tau_j) \text{QS}_\tau [ F^{-1}(\tau), y ]
\end{align}
where $\tau_j = \frac{ j }{ J }$

\section{Suggested threshold weighting functions} \label{sec:weighting}
This comes from p. 415 - Table 4

\begin{align}
    & \text{Brier Scores} && \text{Quantile Scores}\\
    \text{Tails:} \qquad & u(q) = 1 - \phi_{a, b}(q) / \phi_{a,b}(a) && v(\tau) = (2 \tau - 1)^2\\
    \text{Right Tail:} \qquad & u(q) = \Phi_{a, b}(q) && v(\tau) = \tau^2\\
    \text{Left Tail:} \qquad & u(q) = 1 - \Phi_{a, b} && v(\tau) = (1 - \tau)^2
\end{align}
where $\phi_{a, b}$ and $\Phi_{a, b}$ are the pdf and cdf of a $N(a, b)$ distribution, and $a$ and $b$ are parameters to fine-tune the weighting function.


\bibliographystyle{rss}
\bibliography{threshold}
\end{document}