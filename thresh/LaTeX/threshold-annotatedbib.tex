\documentclass[11pt]{article}
\usepackage{amssymb, amsthm, amsmath}
\usepackage{bm}
\usepackage{graphicx}
\usepackage[authoryear]{natbib}
\usepackage{bm}
\usepackage{verbatim}
\usepackage{lineno}
\usepackage{times}
\usepackage{soul}
\usepackage{color}

\usepackage[left=1in,top=1in,right=1in]{geometry}
\input{mycommands.sty}

\pdfpageheight 11in
\pdfpagewidth 8.5in
\linespread{1.0}

\begin{document}\linenumbers
\citep{Gneiting2011}

p. 415 (Table 4) gives suggested threshold weighting for Brier and Quantile scores

\section{Continuous Ranked Probability Scores (CRPS)}
\begin{itemize}
    \item For each of the CRPS versions, they still integrate over all $q$ and $\tau$
    \item For these to be strictly proper, we need a finite first moment ($\xi < 1$)
\end{itemize}


\subsection{Brier scores}
The Brier probability scores is given by 
\begin{align}
    \text{PS}[ F(q), I(y \le q) ] = [ F(q) - I(y \le q) ]^2
\end{align}
where $q$ is a quantile of interest.

The CRPS with Brier score is
\begin{align}
    S(f, y) = \int_{-\infty}^{\infty} \text{PS}[F(q), I(y \le q) ] u(q) \text{d}q
\end{align}
where $u(q)$ is a weighting function (see \ref{sec:weighting}).

Discretized (integral w.r.t. a discrete Stieltjes measure)
\begin{align}
    S(f, y) = \frac{ q_u - q_\ell }{ I - 1 } \sum_{i = 1}^I u(q_i) \text{PS}[ F(q), I(y \le q) ] 
\end{align}
where $(q_\ell, q_u)$ is the range of interest

\subsection{Quantile scores}
The quantile score is given by
\begin{align}
    \text{QS}_\tau [ q(\tau), y ] = 2 \{ I[y \le q(\tau) ] \} [q(\tau) - y]
\end{align}
where $q(\tau)$ is the $\tau$th quantile of the density.

The CRPS with quantile score is
\begin{align}
    S(f, y) = \int_{0}^{1} \text{QS}_\tau (q(\tau), y) v(\tau) \text{d}\tau
\end{align}
where $v(\tau)$ is a weighting function (see \ref{sec:weighting}).

Discretized (integral w.r.t. a discrete Stieltjes measure)
\begin{align}
    S(f, y) = \frac{ 1 }{ J - 1} \sum_{ j = 1 }^{ J-1 } v(\tau_j) \text{QS}_\tau [ F^{-1}(\tau), y ]
\end{align}
where $\tau_j = \frac{ j }{ J }$

\section{Suggested threshold weighting functions} \label{sec:weighting}
This comes from p. 415 - Table 4

\begin{align}
    & \text{Brier Scores} && \text{Quantile Scores}\\
    \text{Tails:} \qquad & u(q) = 1 - \phi_{a, b}(q) / \phi_{a,b}(a) && v(\tau) = (2 \tau - 1)^2\\
    \text{Right Tail:} \qquad & u(q) = \Phi_{a, b}(q) && v(\tau) = \tau^2\\
    \text{Left Tail:} \qquad & u(q) = 1 - \Phi_{a, b} && v(\tau) = (1 - \tau)^2
\end{align}
where $\phi_{a, b}$ and $\Phi_{a, b}$ are the pdf and cdf of a $N(a, b)$ distribution, and $a$ and $b$ are parameters to fine-tune the weighting function.


\bibliographystyle{rss}
\bibliography{threshold}
\end{document}