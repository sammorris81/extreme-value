\documentclass[11pt]{article}
\usepackage{amssymb, amsthm, amsmath}
\usepackage{bm}
\usepackage{graphicx}
\usepackage[authoryear]{natbib}
\usepackage{bm}
\usepackage{verbatim}
\usepackage{lineno}
\usepackage{times}
\usepackage{soul}
\usepackage{color}

\usepackage[left=1in,top=1in,right=1in]{geometry}
\pdfpageheight 11in
\pdfpagewidth 8.5in
\linespread{1.0}
\newcommand{\btheta}{ \mbox{\boldmath $\theta$}}
\newcommand{\bmu}{ \mbox{\boldmath $\mu$}}
\newcommand{\balpha}{ \mbox{\boldmath $\alpha$}}
\newcommand{\bbeta}{ \mbox{\boldmath $\beta$}}
\newcommand{\bdelta}{ \mbox{\boldmath $\delta$}}
\newcommand{\blambda}{ \mbox{\boldmath $\lambda$}}
\newcommand{\bgamma}{ \mbox{\boldmath $\gamma$}}
\newcommand{\brho}{ \mbox{\boldmath $\rho$}}
\newcommand{\bpsi}{ \mbox{\boldmath $\psi$}}
\newcommand{\bepsilon}{ \mbox{\boldmath $\epsilon$}}
\newcommand{\bomega}{ \mbox{\boldmath $\omega$}}
\newcommand{\bOmega}{ \mbox{\boldmath $\Omega$}}
\newcommand{\bDelta}{ \mbox{\boldmath $\Delta$}}
\newcommand{\bSigma}{ \mbox{\boldmath $\Sigma$}}
\newcommand{\bPsi}{\mbox{\boldmath $\Psi$}}
\newcommand{\bOne}{\mbox{\boldmath $1$}}
\newcommand{\omu}{\overline{\mu}}
\newcommand{\oSigma}{\overline{\Sigma}}
\newcommand{\Yt}{{\tilde Y}}
\newcommand{\bA}{ \mbox{\bf A}}
\newcommand{\bP}{ \mbox{\bf P}}
\newcommand{\bx}{ \mbox{\bf x}}
\newcommand{\bX}{ \mbox{\bf X}}
\newcommand{\bB}{ \mbox{\bf B}}
\newcommand{\bZ}{ \mbox{\bf Z}}
\newcommand{\by}{ \mbox{\bf y}}
\newcommand{\bY}{ \mbox{\bf Y}}
\newcommand{\bz}{ \mbox{\bf z}}
\newcommand{\bh}{ \mbox{\bf h}}
\newcommand{\br}{ \mbox{\bf r}}
\newcommand{\bt}{ \mbox{\bf t}}
\newcommand{\bs}{ \mbox{\bf s}}
\newcommand{\bb}{ \mbox{\bf b}}
\newcommand{\bL}{ \mbox{\bf L}}
\newcommand{\bu}{ \mbox{\bf u}}
\newcommand{\bv}{ \mbox{\bf v}}
\newcommand{\bV}{ \mbox{\bf V}}
\newcommand{\bW}{ \mbox{\bf W}}
\newcommand{\bG}{ \mbox{\bf G}}
\newcommand{\bH}{ \mbox{\bf H}}
\newcommand{\bw}{ \mbox{\bf w}}
\newcommand{\bo}{ \mbox{\bf o}}
\newcommand{\bfe}{ \mbox{\bf e}}
\newcommand{\iid}{\stackrel{iid}{\sim}}
\newcommand{\indep}{\stackrel{indep}{\sim}}
\newcommand{\calR}{{\cal R}}
\newcommand{\calG}{{\cal G}}
\newcommand{\calD}{{\cal D}}
\newcommand{\calS}{{\cal S}}
\newcommand{\calB}{{\cal B}}
\newcommand{\calA}{{\cal A}}
\newcommand{\calT}{{\cal T}}
\newcommand{\calO}{{\cal O}}
\newcommand{\argmax}{{\mathop{\rm arg\, max}}}
\newcommand{\argmin}{{\mathop{\rm arg\, min}}}
\newcommand{\Frechet}{\mbox{Fr$\acute{\mbox{e}}$chet }}
\newcommand{\Matern}{\mbox{Mat$\acute{\mbox{e}}$rn }}
\newcommand{\ballunion}{B_a(\bs_1) \cup B_b(\bs_2) }

\newcommand{\beq}{ \begin{equation}}
\newcommand{\eeq}{ \end{equation}}
\newcommand{\beqn}{ \begin{eqnarray}}
\newcommand{\eeqn}{ \end{eqnarray}}


\begin{document}\linenumbers

\begin{center}
{\Large {\bf A new spatial model for points above a threshold}}\\
\today
\end{center}

\section{Introduction}\label{s:intro}

\section{Statistical model}\label{s:model}

Let $Y_t(\bs)\in \calR$ be the observed value at location $\bs$ on day $t$.  To avoid bias in estimating tail parameters, we model the thresholded data
\beq\label{Yt}
  \Yt_t(\bs) = \left\{
          \begin{array}{ll}
            Y_t(\bs) & Y_t(\bs)>T \\
            T & Y_t(\bs)\le T
          \end{array}
        \right.
\eeq
where $T$ is a pre-specified threshold.   

We first specify a model for the complete data, $Y_t(\bs)$, and then study the induced model for thresholded data, $\Yt_t(\bs)$.  
The full data model is given in Section \ref{s:model} assuming a skew normal distribution with a different variance each day.
Computationally, the values below the threshold are updated using standard Bayesian missing data methods as described in Section \ref{s:comp}.
The skew normal representation is from \citep{Minozzo2012} and is the sum of a normal and half-normal random variable.
\subsection{Half-normal distribution}\label{s:hn}
Let $u = |z|$ where $Z \sim N(\mu, \sigma^2)$.
Specifically, we consider the case where $\mu = 0$. Then $U$ follows a half-normal distribution which we denote as $U \sim HN(0, 1)$, and the density is given by 
\begin{align}
  f_U(u) = \frac{ \sqrt{2} }{ \sqrt{\pi \sigma^2} } \exp \left( - \frac{ u^2 }{ 2 \sigma^2 } \right) I(u > 0)
\end{align}
When $\mu = 0$, the half-normal distribution is also equivalent to a $N_{(0, \infty)}(0, \sigma^2)$ where $N_{(a, b)}(\mu, \sigma^2)$ represents a normal distribution with mean $\mu$ and standard deviation $\sigma$ that has been truncated below at $a$ and above at $b$.

\subsection{Complete data}\label{s:model}
Consider a skew Gaussian spatial process
\begin{align} \label{eq:fullmodel}
  Y_t(\bs) &= X_t(\bs) \beta + \sigma_1 z_t(\bs) + v_t(\bs)
\end{align}
where $z_t(\bs) = z_{tl}$ if $s \in P_{tl}$ where $P_{t1}, \ldots, P_{tL}$ form a partition, and $z_{tl} \iid N_{(0, \infty)}(0, 1)$, $\sigma_1 \in \calR$, $\sigma_2, \sigma_0 \in \calR^+$, and $v_t(\bs)$ is a spatial Gaussian process with mean zero and variance $\sigma_{tl}^2$.
It can be shown \citep{Zhang2010} that $Y_t(\bs)$ follows a skew normal distribution with skewness parameter $\alpha = \frac{ \sigma_1 }{ \sigma_{tl} }$.
We can then reexpress the model in (\ref{eq:fullmodel}) as
\begin{align}
  Y_t(\bs) &= X_t(\bs) \beta + \alpha z_t(\bs) + v_t(\bs)
\end{align}
where $z_t(\bs) = z_{tl}$, $z_{tl} \iid N_{(0, \infty)}(0, \sigma_{tl}^2)$, and $v_t(\bs)$ is defined as before.

We model this with a Bayesian hierarchical model as follows.
Let $w_{t1}, \ldots, w_{tL}$ be partition centers so that
\begin{align*}
  P_{tl} = \{\bs_t : l = \argmin_k || \bs_t - w_{tk} ||\}.
\end{align*}
Then
\begin{align}
    Y_t(\bs) \mid \Theta, z_{t1}, \ldots, z_{tL} &= X_t(\bs) \beta + \alpha z_t(\bs) + v_t(\bs) \label{eq:hier}\\
    z_{tl}(\bs) \mid \Theta &\sim N_{(0, \infty)}(0, \sigma^2_{tl})\\
    v_t(\bs) \mid \Theta &\sim \Matern(0, \Sigma)\\
    \sigma^2_{tl} &\iid IG(\alpha, \beta)\\
    \alpha &\sim N(0, 10)\\
    w_{tk} &\sim Unif(\calD)
\end{align}
where $\Theta = \{w_{t1}, \ldots, w_{tL}, \beta, \sigma_t, \delta, \rho, \nu \}$; $l = \argmin_k ||\bs - w_k ||$; $\Sigma_t$ is a \Matern covariance matrix with variance $\sigma_{tl}^2 (1 - \delta^2)$, spatial range $\rho$ and smoothness $\nu$; and $\calD$ is the spatial domain of interest.

\section{Computation}\label{s:comp}
The MCMC for this model is fairly straightforward.
First, we impute values below the threshold.
Then, we update $\Theta$ using random walk MH or Gibbs sampling when appropriate.
Finally, we make spatial predictions.
Each requires the joint distribution for the complete data given $\Theta$.
As defined in \ref{eq:hier}, the distribution of $Y_t(\bs) \mid \Theta$ is the usual multivariate normal distribution with a \Matern spatial covariance structure.

\subsection{Imputation}\label{s:impute}
We can use Gibbs sampling to update $\Yt_t(\bs)$ for observations that are below $T$, the thresholded value. Given $\Theta$, $Y_t(\bs)$ has truncated normal full conditional with these parameter values.
So we sample $Y_t(\bs) \sim N_{(-\infty, T)}$

\subsection{Parameter updates}\label{s:params}
To update $\Theta$ given the current value of the complete data $\bY_1, \ldots, \bY_T$, we use a standard Gibbs updates for all parameters except for the knot locations which are done using a Metropolis update.
See Appendix A.1 for details regarding Gibbs sampling.

\subsection{Spatial prediction}\label{s:pred}
Given $\bY_t$ the usual Kriging equations give the predictive distribution for $Y_t(\bs^*)$ at prediction location $(\bs^*)$


\section{Data analysis}\label{s:analysis}


\section{Conclusions}\label{s:con}

\section*{Acknowledgments}

\section*{Appendix A.1: Posterior distributions}



%\subsubsection*{Conditional posterior of $U | Y$}\label{s:condu}
% Let $Y_i | U \sim \mbox{N}(U, \sigma^2)$, $i = 1, \ldots, n$, let $\tau = 1 / \sigma^2$, and let $\pi(U) \propto \exp \left\{ -\frac{ u^2 \theta }{ 2 } \right\}$. 
% Then the conditional posterior of $U \mid \ldots$ is 
% \begin{align}
%   \pi (U \mid \ldots) & \propto \exp \left\{ -\frac{ u^2 \theta }{ 2 } \right\} \exp \left\{ - \sum_{i = 1 }^n\frac{ \tau (y_i - u)^2 }{ 2 } \right\} \nonumber \\
%     & \propto \exp \left\{ -\frac{ 1 }{ 2 } \left[ u^2 \theta + \sum_{i=1 }^n\tau (y_i^2 - 2y_iu + u^2) \right] \right\} \nonumber \\
%     & \propto \exp \left\{ - \frac{ 1 }{ 2 }\left( u - \frac{ \tau \sum_{i=1}^n y_i }{ \theta + n \tau } \right)^2 \left( \theta + n \tau \right) \right\} \nonumber\\
%     & \propto \mbox{HN}(\xi^*, \theta^*) \label{eq:condu}
% \end{align}
% where 
% \begin{align*}
%   \xi^* &= \frac{ \tau \sum_{i=1}^n y_i }{ \theta + n \tau }\\
%   \theta^* &= \theta + n \tauå
% \end{align*}

\subsubsection*{Conditional posterior of $z_{tl} \mid \ldots $}\label{s:mvcondu}
For simplicity, drop the subscript $t$ and define 
\begin{align*}
R(\bs) = \left\{ 
    \begin{array}{ll}
        Y(\bs) - X(\bs) \beta &s \in P_l\\[1em]
        Y(\bs) - X(\bs) \beta - \delta z(\bs) \qquad & s \notin P_l
    \end{array} 
\right.
\end{align*}
Let 
\begin{align*}
    R_1 &= \text{the vector of } R(\bs) \text{ for } s \in P_l \\
    R_2 &= \text{the vector of } R(\bs) \text{ for } s \notin P_l \\
    \Omega &= \Sigma^{-1}.
\end{align*}
Then
\begin{align*}
    \pi(z_l | \ldots) &\propto \exp \left\{ -\frac{ 1 }{ 2 \sigma^2 } \left[ \frac{ 1 }{ (1 - \delta^2)}
        \left( \begin{array}{c}
            R_1 - \delta z_l \bOne\\
            R_2
        \end{array} \right)^T
        \left( \begin{array}{cc}
            \Omega_{11} & \Omega_{12}\\
            \Omega_{21} & \Omega_{22}
        \end{array} \right)
        \left( \begin{array}{c}
            R_1 - \delta z_l \bOne\\
            R_2
        \end{array} \right)
        +  z_l^2 \right]
    \right\} I(z_l > 0) \\
        &\propto \exp \left\{ -\frac{ 1 }{ 2 \sigma^2 } \left[ \Lambda_z z_l^2 - 2 \mu_z z_l \right] \right\} I(z_l > 0)
\end{align*}
where
\begin{align*}
    \mu_z &= \frac{ \delta}{(1 - \delta^2)} ( R_1^T \Omega_{11} + R_2^T \Omega_{21} )\bOne\\
    \Lambda_z &= \frac{\delta^2 \bOne^T \Omega_{11} \bOne }{ (1 - \delta^2) } + 1.
\end{align*}
Then $Z_l | \ldots \sim N_{(0, \infty)} (\Lambda_z^{-1} \mu_z, \sigma^2 \Lambda_z^{-1})$
\subsection*{Conditional posterior of $\beta \mid \ldots$}\label{s:betapost}
Let $\beta \sim \mbox{N}_{p}(0, \Lambda_0)$ where $\Lambda_0$ is a precision matrix. Then 
\begin{align*}
    \pi(\beta \mid \ldots) & \propto \exp \left\{ - \frac{ 1 }{ 2 } \beta^T \Lambda_0 \beta - \sum_{t = 1 }^T \frac{ 1 }{ 2 } [\bY_t(\bs) - X_t(\bs) \beta - \sigma \delta |u_t|]^T \Sigma^{-1} [\bY_t(\bs) - X_t(\bs) \beta - u^*_t] \right\}\\
     & \propto \exp \left\{ -\frac{ 1 }{ 2 } \left[ \beta^T \Lambda_p \beta  - \sum_{ t = 1 }^T 2 [ \beta^T X_t(\bs) \Sigma^{-1} (\bY_t(\bs) + u^*_t )] \right] \right\}\\
     & \propto \mbox{N}_p ( \mu_p , \Lambda_p)
\end{align*}
where
\begin{align*}
    \mu_p &= \Lambda_p^{-1} \left[ X_t(\bs)^T \Sigma^{-1} (\bY_t(\bs) + u^*_t) \right]\\
    \Lambda_p &= \left (\Lambda_0 + \sum_{ t = 1 }^{ T} X_t(\bs)^T \Sigma^{-1} X_t(\bs) \right)
\end{align*}
and $\Lambda_p$ is a precision matrix.
\subsection*{Conditional posterior of $\sigma^2 \mid \ldots$}\label{s:sigpost}
In the case where $L = 1$, then $\sigma^2$ has a conjugate posterior distribution. 
Let $\sigma_t^2 \iid \mbox{IG}(\alpha_0, \beta_0)$. For simplicity, drop the subscript $t$. Then
\begin{align*}
    \pi(\sigma^2 \mid \ldots) & \propto (\sigma^2)^{ -\alpha_0 - 1 / 2 - n / 2 - 1} \exp \left\{ -\frac{\beta_0}{\sigma^2} - \frac{ z^2 }{2 \sigma^2} - \frac{ (\bY - \bmu)^T \Sigma^{-1} (\bY - \bmu) }{2 \sigma^2} \right\} \\
    & \propto (\sigma^2)^{ -\alpha_0 - 1 / 2 - n / 2 - 1} \exp \left\{ - \frac{ 1 }{ \sigma^2 } \left[\beta_0 + \frac{ z^2 }{ 2 } + \frac{ 1 }{ 2 } (\bY - \bmu)^T \Sigma^{-1} (\bY - \mu) \right] \right\} \\
    & \propto \mbox{IG} (\alpha^*, \beta^*)
\end{align*}
where
\begin{align*}
    \alpha^* &= \alpha_0 + \frac{1}{2} + \frac{n}{2} \\
    \beta^* &= \beta_0 + \frac{ z^2 }{ 2 } + \frac{ 1 }{ 2 }(\bY - \bmu)^T \Sigma^{-1} (\bY - \bmu).
\end{align*}
In the case that $L > 1$, a random walk Metropolis Hastings step will be used to update $\sigma^2_{lt}$.
\subsection*{Conditional posterior of $\alpha \mid \ldots$}\label{s:alphapost}
Let $\alpha \sim N(0, \tau_\alpha)$ where $\tau_\alpha$ is a precision term. Then
\begin{align*}
  \pi(\alpha \mid \ldots) &\propto \exp \left\{ -\frac{ 1 }{ 2 } \tau_\alpha \alpha^2 + \sum_{ t = 1 }^T \frac{ 1 }{ 2 } [\bY_t - X_t\beta - \alpha |z_t|]^T \Omega [\bY_t - X_t\beta - \alpha |z_t|] \right\} \\
      &\propto \exp \left\{ -\frac{ 1 }{ 2 } [\alpha^2(\tau_\alpha + \sum_{t = 1 }^T |z_t|^T \Omega |z_t|^T) - 2 \alpha \sum_{t=1}^T[|z_t|^T \Omega (\bY_t - X_t \beta) \right\}\\
      &\propto \exp \left\{ -\frac{ 1 }{ 2 } [\tau_\alpha^* \alpha^2 - 2 \mu_\alpha] \right\}
\end{align*}
where
\begin{align*}
  \mu_\alpha &= \sum_{t = 1}^T |z_t|^T \Omega (\bY_t - X_t \beta)\\
  \tau_\alpha^* &= t_\alpha + \sum_{t=1}^T |z_t|^T \Omega |z_t|.
\end{align*}
Then $\alpha \mid \ldots \sim N(\tau^{*-1}_\alpha \mu_\alpha, \tau^{*-1}_\alpha) $

\section*{Appendix A.2: MCMC Details}

\subsection*{Priors}



\bibliographystyle{rss}
\bibliography{skewnormal}

\end{document}

