\documentclass[11pt]{article}
\usepackage{amssymb, amsthm, amsmath}
\usepackage{bm}
\usepackage{graphicx}
\usepackage[authoryear]{natbib}
\usepackage{bm}
\usepackage{verbatim}
\usepackage{lineno}
\usepackage{times}
\usepackage{soul}
\usepackage{color}

\usepackage[left=1in,top=1in,right=1in]{geometry}
\pdfpageheight 11in
\pdfpagewidth 8.5in
\linespread{1.0}
\newcommand{\btheta}{ \mbox{\boldmath $\theta$}}
\newcommand{\bmu}{ \mbox{\boldmath $\mu$}}
\newcommand{\balpha}{ \mbox{\boldmath $\alpha$}}
\newcommand{\bbeta}{ \mbox{\boldmath $\beta$}}
\newcommand{\bdelta}{ \mbox{\boldmath $\delta$}}
\newcommand{\blambda}{ \mbox{\boldmath $\lambda$}}
\newcommand{\bgamma}{ \mbox{\boldmath $\gamma$}}
\newcommand{\brho}{ \mbox{\boldmath $\rho$}}
\newcommand{\bpsi}{ \mbox{\boldmath $\psi$}}
\newcommand{\bepsilon}{ \mbox{\boldmath $\epsilon$}}
\newcommand{\bomega}{ \mbox{\boldmath $\omega$}}
\newcommand{\bOmega}{ \mbox{\boldmath $\Omega$}}
\newcommand{\bDelta}{ \mbox{\boldmath $\Delta$}}
\newcommand{\bSigma}{ \mbox{\boldmath $\Sigma$}}
\newcommand{\bPsi}{\mbox{\boldmath $\Psi$}}
\newcommand{\bOne}{\mbox{\boldmath $1$}}
\newcommand{\omu}{\overline{\mu}}
\newcommand{\oSigma}{\overline{\Sigma}}
\newcommand{\Yt}{{\tilde Y}}
\newcommand{\bA}{ \mbox{\bf A}}
\newcommand{\bP}{ \mbox{\bf P}}
\newcommand{\bx}{ \mbox{\bf x}}
\newcommand{\bX}{ \mbox{\bf X}}
\newcommand{\bB}{ \mbox{\bf B}}
\newcommand{\bZ}{ \mbox{\bf Z}}
\newcommand{\by}{ \mbox{\bf y}}
\newcommand{\bY}{ \mbox{\bf Y}}
\newcommand{\bz}{ \mbox{\bf z}}
\newcommand{\bh}{ \mbox{\bf h}}
\newcommand{\br}{ \mbox{\bf r}}
\newcommand{\bt}{ \mbox{\bf t}}
\newcommand{\bs}{ \mbox{\bf s}}
\newcommand{\bb}{ \mbox{\bf b}}
\newcommand{\bL}{ \mbox{\bf L}}
\newcommand{\bu}{ \mbox{\bf u}}
\newcommand{\bv}{ \mbox{\bf v}}
\newcommand{\bV}{ \mbox{\bf V}}
\newcommand{\bW}{ \mbox{\bf W}}
\newcommand{\bG}{ \mbox{\bf G}}
\newcommand{\bH}{ \mbox{\bf H}}
\newcommand{\bw}{ \mbox{\bf w}}
\newcommand{\bo}{ \mbox{\bf o}}
\newcommand{\bfe}{ \mbox{\bf e}}
\newcommand{\iid}{\stackrel{iid}{\sim}}
\newcommand{\indep}{\stackrel{indep}{\sim}}
\newcommand{\calR}{{\cal R}}
\newcommand{\calG}{{\cal G}}
\newcommand{\calD}{{\cal D}}
\newcommand{\calS}{{\cal S}}
\newcommand{\calB}{{\cal B}}
\newcommand{\calA}{{\cal A}}
\newcommand{\calT}{{\cal T}}
\newcommand{\calO}{{\cal O}}
\newcommand{\argmax}{{\mathop{\rm arg\, max}}}
\newcommand{\argmin}{{\mathop{\rm arg\, min}}}
\newcommand{\Frechet}{\mbox{Fr$\acute{\mbox{e}}$chet }}
\newcommand{\Matern}{\mbox{Mat$\acute{\mbox{e}}$rn }}
\newcommand{\ballunion}{B_a(\bs_1) \cup B_b(\bs_2) }

\newcommand{\beq}{ \begin{equation}}
\newcommand{\eeq}{ \end{equation}}
\newcommand{\beqn}{ \begin{eqnarray}}
\newcommand{\eeqn}{ \end{eqnarray}}


\begin{document}\linenumbers

\begin{center}
{\Large {\bf A new spatial model for points above a threshold}}\\
\today
\end{center}

\section{Introduction}\label{s:intro}

\section{Statistical model}\label{s:model}

Let $Y_t(\bs)\in \calR$ be the observed value at location $\bs$ on day $t$.  To avoid bias in estimating tail parameters, we model the thresholded data
\beq\label{Yt}
  \Yt_t(\bs) = \left\{
          \begin{array}{ll}
            Y_t(\bs) & Y_t(\bs)>T \\
            T & Y_t(\bs)\le T
          \end{array}
        \right.
\eeq
where $T$ is a pre-specified threshold.   

We first specify a model for the complete data, $Y_t(\bs)$, and then study the induced model for thresholded data, $\Yt_t(\bs)$.  
The full data model is given in Section \ref{s:model} assuming a multivariate normal distribution with a different variance each day.
Computationally, the values below the threshold are updated using standard Bayesian missing data methods as described in Section \ref{s:comp}.

\subsection{Complete data}\label{s:model}
Consider the spatial process
\begin{align} \label{fullmodel}
  Y_t(\bs) &= X_t(\bs) \beta + e_t(\bs)\\
  e_t(\bs) &= \sigma \delta | u_t(\bs) | + v_t(\bs)
\end{align}
where $u_t(\bs) = u_{tl}$ if $s \in P_{tl}$ where $P_{t1}, \ldots, P_{tL}$ form a partition, and $u_{tl} \iid N(0, 1)$, $\delta \in (-1, 1)$ controls skew, and $v_t(\bs)$ is a spatial process with mean zero and variance $\sigma^2(1 - \delta^2)$.
Then $Y_t(\bs)$ is skew normal within each partition \citep{Minozzo2012}.
We model this with a Bayesian hierarchical model as follows.
Let $w_{t1}, \ldots, w_{tL}$ be partition centers so that $P_{tl}$ includes all spatial locations $\bs$ that are within the partition.
Then
\begin{align} \label{eq:hier}
    Y_t(\bs) \mid \Theta &= \mu_t(\bs) + v_t(\bs) \\
    \mu_t(\bs) &= X_t(\bs) \beta + \sigma \delta | u_{tl} |
\end{align}
where $l = \argmin_j ||\bs - w_j ||$ and $\Theta = \{ u_{t1}, \ldots, u_{tL}, w_{t1}, \ldots, w_{tL}, \beta, \rho, \nu, \sigma \}$ are the random effects, knot locations, and parameters for the mean, and spatial covariance.


\section{Computation}\label{s:comp}
The MCMC for this model is fairly straightforward.
First, we impute values below the threshold.
Then, we update $\Theta$ using random walk MH or Gibbs sampling when appropriate.
Finally, we make spatial predictions.
Each requires the joint distribution for the complete data given $\Theta$.
As defined in \ref{eq:hier}, the distribution of $Y_t(\bs) \mid \Theta$ is the usual multivariate normal distribution with a \Matern spatial covariance structure.

\subsection{Imputation}\label{s:impute}
We can use Gibbs sampling to update $\Yt_t(\bs)$ for observations that are below $T$, the thresholded value. Given $\Theta$, $Y_t(\bs)$ has truncated normal full conditional with these parameter values.
So we sample $Y_t(\bs) \sim \text{TN}_{(-\infty, T)}$

\subsection{Parameter updates}\label{s:params}
To update $\Theta$ given the current value of the complete data $\bY_1, \ldots, \bY_T$, we use a standard Metropolis update.

\subsection{Spatial prediction}\label{s:pred}
Given $\bY_t$ the usual Kriging equations give the predictive distribution for $Y_t(\bs^*)$ at prediction location $(\bs^*)$


\section{Data analysis}\label{s:analysis}


\section{Conclusions}\label{s:con}

\section*{Acknowledgments}

\section*{Appendix A.1: MCMC Details}

\subsection*{Priors}
For a given day
\beqn
	r_{j} &\iid& \mbox{IG}(\xi_r, \sigma_r)\nonumber\\ 
	\sigma_r &\sim& \mbox{Gamma}(0.1, 0.1)\nonumber\\
	\xi_r &\sim& \mbox{Discrete Uniform}(0.5, 30)\nonumber\\
	\bv_{j} &\iid& \mbox{Uniform}(\calD)\nonumber\\
	\mu(\bs) &\sim& \mbox{MVN}(0, \mbox{diag}(10))\nonumber\\
	\log(\rho) &\sim& \mbox{N}(0, 10)\nonumber\\
	\log(\nu) &\sim& \mbox{N}(-1, 1)\nonumber\\
	\alpha &\sim& \mbox{Unif}(0, 1)\nonumber
\eeqn
where $v_j$ are the locations of the spatial knots over $\calD$, $\alpha$ is a parameter controlling the proportion of $r^2_{j}$ that is attributed to the nugget and partial sill. 
If $\alpha = 0$, then $r^2_{j}$ can be entirely attributed to the nugget effect, and if $\alpha = 1$, then $r^2_{tj}$ can be entirely attributed to the partial sill.
We use Gibbs sampling for $r_j, \sigma_r$, and $\mu(\bs)$. 
All other parameters are sampled using a random-walk Metropolis Hastings algorithm.

\bibliographystyle{rss}
\bibliography{Schlather}

\end{document}

