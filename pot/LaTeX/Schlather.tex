\documentclass[11pt]{article}
\usepackage{amssymb, amsthm, amsmath}
\usepackage{bm}
\usepackage{graphicx}
\usepackage[authoryear]{natbib}
\usepackage{bm}
\usepackage{verbatim}
\usepackage{lineno}
\usepackage{times}
\usepackage{soul}
\usepackage{color}

\usepackage[left=1in,top=1in,right=1in]{geometry}
\pdfpageheight 11in
\pdfpagewidth 8.5in
\linespread{1.0}
\newcommand{\btheta}{ \mbox{\boldmath $\theta$}}
\newcommand{\bmu}{ \mbox{\boldmath $\mu$}}
\newcommand{\balpha}{ \mbox{\boldmath $\alpha$}}
\newcommand{\bbeta}{ \mbox{\boldmath $\beta$}}
\newcommand{\bdelta}{ \mbox{\boldmath $\delta$}}
\newcommand{\blambda}{ \mbox{\boldmath $\lambda$}}
\newcommand{\bgamma}{ \mbox{\boldmath $\gamma$}}
\newcommand{\brho}{ \mbox{\boldmath $\rho$}}
\newcommand{\bpsi}{ \mbox{\boldmath $\psi$}}
\newcommand{\bepsilon}{ \mbox{\boldmath $\epsilon$}}
\newcommand{\bomega}{ \mbox{\boldmath $\omega$}}
\newcommand{\bOmega}{ \mbox{\boldmath $\Omega$}}
\newcommand{\bDelta}{ \mbox{\boldmath $\Delta$}}
\newcommand{\bSigma}{ \mbox{\boldmath $\Sigma$}}
\newcommand{\bPsi}{\mbox{\boldmath $\Psi$}}
\newcommand{\bOne}{\mbox{\boldmath $1$}}
\newcommand{\omu}{\overline{\mu}}
\newcommand{\oSigma}{\overline{\Sigma}}
\newcommand{\Yt}{{\tilde Y}}
\newcommand{\bA}{ \mbox{\bf A}}
\newcommand{\bP}{ \mbox{\bf P}}
\newcommand{\bx}{ \mbox{\bf x}}
\newcommand{\bX}{ \mbox{\bf X}}
\newcommand{\bB}{ \mbox{\bf B}}
\newcommand{\bZ}{ \mbox{\bf Z}}
\newcommand{\by}{ \mbox{\bf y}}
\newcommand{\bY}{ \mbox{\bf Y}}
\newcommand{\bz}{ \mbox{\bf z}}
\newcommand{\bh}{ \mbox{\bf h}}
\newcommand{\br}{ \mbox{\bf r}}
\newcommand{\bt}{ \mbox{\bf t}}
\newcommand{\bs}{ \mbox{\bf s}}
\newcommand{\bb}{ \mbox{\bf b}}
\newcommand{\bL}{ \mbox{\bf L}}
\newcommand{\bu}{ \mbox{\bf u}}
\newcommand{\bv}{ \mbox{\bf v}}
\newcommand{\bV}{ \mbox{\bf V}}
\newcommand{\bW}{ \mbox{\bf W}}
\newcommand{\bG}{ \mbox{\bf G}}
\newcommand{\bH}{ \mbox{\bf H}}
\newcommand{\bw}{ \mbox{\bf w}}
\newcommand{\bo}{ \mbox{\bf o}}
\newcommand{\bfe}{ \mbox{\bf e}}
\newcommand{\iid}{\stackrel{iid}{\sim}}
\newcommand{\indep}{\stackrel{indep}{\sim}}
\newcommand{\calR}{{\cal R}}
\newcommand{\calG}{{\cal G}}
\newcommand{\calD}{{\cal D}}
\newcommand{\calS}{{\cal S}}
\newcommand{\calB}{{\cal B}}
\newcommand{\calA}{{\cal A}}
\newcommand{\calT}{{\cal T}}
\newcommand{\calO}{{\cal O}}
\newcommand{\argmax}{{\mathop{\rm arg\, max}}}
\newcommand{\argmin}{{\mathop{\rm arg\, min}}}
\newcommand{\Frechet}{\mbox{Fr$\acute{\mbox{e}}$chet }}
\newcommand{\Matern}{\mbox{Mat$\acute{\mbox{e}}$rn }}
\newcommand{\ballunion}{B_a(\bs_1) \cup B_b(\bs_2) }

\newcommand{\beq}{ \begin{equation}}
\newcommand{\eeq}{ \end{equation}}
\newcommand{\beqn}{ \begin{eqnarray}}
\newcommand{\eeqn}{ \end{eqnarray}}


\begin{document}\linenumbers

\begin{center}
{\Large {\bf A new spatial model for points above a threshold}}\\
\today
\end{center}

\section{Introduction}\label{s:intro}

\section{Statistical model}\label{s:model}

Let $Y_t(\bs)\in \calR$ be the observed value at location $\bs$ on day $t$.  To avoid bias in estimating tail parameters, we model the thresholded data
\beq\label{Yt}
  \Yt_t(\bs) = \left\{
          \begin{array}{ll}
            Y_t(\bs) & Y_t(\bs)>T \\
            T & Y_t(\bs)\le T
          \end{array}
        \right.
\eeq
where $T$ is a pre-specified threshold.   

We first specify a model for the complete data, $Y_t(\bs)$, and then study the induced model for thresholded data, $\Yt_t(\bs)$.  
The full data model is given in Section \ref{s:model} assuming a multivariate normal distribution with a different variance each day.
Computationally, the values below the threshold are updated using standard Bayesian missing data methods as described in Section \ref{s:comp}.

\subsection{Complete data}\label{s:model}
Consider the spatial process
\begin{align} \label{eq:fullmodel}
  Y_t(\bs) &= X_t(\bs) \beta + e_t(\bs)\\
  e_t(\bs) &= \sigma \delta | u_t(\bs) | + v_t(\bs)
\end{align}
where $u_t(\bs) = u_{tl}$ if $s \in P_{tl}$ where $P_{t1}, \ldots, P_{tL}$ form a partition, and $u_{tl} \iid N(0, 1)$, $\delta \in (-1, 1)$ controls skew, and $v_t(\bs)$ is a spatial process with mean zero and variance $\sigma^2(1 - \delta^2)$.
Then $Y_t(\bs)$ is skew normal within each partition \citep{Minozzo2012}.
We model this with a Bayesian hierarchical model as follows.
Let $w_{t1}, \ldots, w_{tL}$ be partition centers so that $P_{tl}$ includes all spatial locations $\bs$ that are within the partition.
Then
\begin{align} \label{eq:hier}
    Y_t(\bs) \mid \Theta &= \mu_t(\bs) + v_t(\bs) \\
    \mu_t(\bs) &= X_t(\bs) \beta + \sigma \delta | u_{tl} |
\end{align}
where $l = \argmin_j ||\bs - w_j ||$ and $\Theta = \{ u_{t1}, \ldots, u_{tL}, w_{t1}, \ldots, w_{tL}, \beta, \rho, \nu, \sigma \}$ are the random effects, knot locations, and parameters for the mean, and spatial covariance.

\section{Computation}\label{s:comp}
The MCMC for this model is fairly straightforward.
First, we impute values below the threshold.
Then, we update $\Theta$ using random walk MH or Gibbs sampling when appropriate.
Finally, we make spatial predictions.
Each requires the joint distribution for the complete data given $\Theta$.
As defined in \ref{eq:hier}, the distribution of $Y_t(\bs) \mid \Theta$ is the usual multivariate normal distribution with a \Matern spatial covariance structure.

\subsection{Imputation}\label{s:impute}
We can use Gibbs sampling to update $\Yt_t(\bs)$ for observations that are below $T$, the thresholded value. Given $\Theta$, $Y_t(\bs)$ has truncated normal full conditional with these parameter values.
So we sample $Y_t(\bs) \sim \text{TN}_{(-\infty, T)}$

\subsection{Parameter updates}\label{s:params}
To update $\Theta$ given the current value of the complete data $\bY_1, \ldots, \bY_T$, we use a standard Gibbs updates for all parameters except for the knot locations which are done using a Metropolis update.
See Appendix A.1 for details regarding Gibbs sampling and $|u_t(\bs)|$.

\subsection{Spatial prediction}\label{s:pred}
Given $\bY_t$ the usual Kriging equations give the predictive distribution for $Y_t(\bs^*)$ at prediction location $(\bs^*)$


\section{Data analysis}\label{s:analysis}


\section{Conclusions}\label{s:con}

\section*{Acknowledgments}

\section*{Appendix A.1: Half-Normal results}

\subsection*{Half-normal}\label{s:hn}
Let $u = \xi + \sqrt{\eta} |x|$ where $X \sim N(0, 1)$.
Then \citet{Wiper2008} show that $U$ follows a half-normal distribution which we shall write as $U \sim \mbox{HN}(\xi, \theta )$ where $\theta = \displaystyle \frac{ 1 }{ \eta }$ is a precision term.
The density is given by 
\begin{align}
  f_U(u) = \frac{ \sqrt{\theta \pi} }{ \sqrt{2} } \exp \left( - \frac{ (u-\xi)^2 \theta }{ 2 } \right), \quad u > \xi.
\end{align}

\subsubsection*{Conditional posterior of $U | Y$}\label{s:condu}
Let $Y_i | U \sim \mbox{N}(U, \sigma^2)$, $i = 1, \ldots, n$, let $\tau = 1 / \sigma^2$, and let $\pi(U) \propto \exp \left\{ -\frac{ u^2 \theta }{ 2 } \right\}$. 
Then the conditional posterior of $U | Y$ is 
\begin{align}
  \pi (U \mid Y) & \propto \exp \left\{ -\frac{ u^2 \theta }{ 2 } \right\} \exp \left\{ - \sum_{i = 1 }^n\frac{ \tau (y_i - u)^2 }{ 2 } \right\} \nonumber \\
    & \propto \exp \left\{ -\frac{ 1 }{ 2 } \left[ u^2 \theta + \sum_{i=1 }^n\tau (y_i^2 - 2y_iu + u^2) \right] \right\} \nonumber \\
    & \propto \exp \left\{ - \frac{ 1 }{ 2 }\left( u - \frac{ \tau \sum_{i=1}^n y_i }{ \theta + n \tau } \right)^2 \left( \theta + n \tau \right) \right\} \nonumber\\
    & \propto \mbox{HN}(\xi^*, \theta^*) \label{eq:condu}
\end{align}
where 
\begin{align*}
  \xi^* &= \frac{ \tau \sum_{i=1}^n y_i }{ \theta + n \tau }\\
  \theta^* &= \theta + n \tau
\end{align*}

\subsubsection*{Conditional posterior of $U_{tl} | \bY_{tl}(\bs)$}\label{s:mvcondu}
Consider a multivariate response $Y_t(\bs)$ as given by (\ref{eq:hier}) with two partitions. 
Then conditioned on the observations in partition 2,  
\begin{align}
  Y_{t1} | Y_{t2} \sim \mbox{N}_{n_1} (\omu, \oSigma)
\end{align}
where $\omu = \mu_1 + \Sigma_{12} \Sigma_{22}^{-1}(y_{t2} - \mu_{2})$, and $\oSigma = \Sigma_{11} - \Sigma_{12} \Sigma_{22}^{-1}\Sigma_{21}$. 
Then conditional posterior of $U_{t1} | \bY_{t1}$ is
\begin{align}
  \pi(U_{t1} | \bY_{t1}) & \propto \exp \left\{ -\frac{ 1 }{ 2 \sigma^2 \delta^2 } u^2 - \frac{ 1 }{ \sigma^2 (1 - \delta^2 ) } \left[ \bY_{t1} - \omu \right]^T \oSigma^{-1} \left[ \bY_{t1} - \omu \right] \right\}\\
  & \propto \exp \left\{ - \frac{1}{2 \sigma^2} \left[ \frac{ 1 }{ \delta^2 } u^2 + \frac{ \sigma^2 \delta^2 }{ (1 - \delta^2) } u^2 \bOne^T \oSigma^{-1} \bOne - 2u\bOne^T \oSigma^{-1} [\bY_{t1} - X_{t1}\beta - \Sigma_{12} \Sigma_{22}^{-1} (\bY_{t2} - \mu_2)] \right]\right\}\\
  & \propto \exp \{ -\frac{ 1 }{ 2 } (u - \xi^*)^2 (\theta^*) \}
\end{align}
where
\begin{align}
  \xi^* &= \frac{ \sigma \delta \bOne^T \oSigma^{-1}\left[ \bY_{t1} - X_{t1}\beta - \Sigma_{12} \Sigma_{22}^{-1} (\bY_t2 - \mu_2)\right]}{\frac{ 1 }{ \delta^2 } + \frac{ \sigma^2 \delta^2 \bOne^T \oSigma^{-1} \bOne}{(1 - \delta^2)}}\\
  \theta^* &= \frac{ 1 }{ \sigma^2 \delta^2 } + \frac{ \delta^2 \bOne^T \oSigma^{-1} \bOne }{ (1 - \delta^2 )}
\end{align}




\section*{Appendix A.2: MCMC Details}

\subsection*{Priors}



\bibliographystyle{rss}
\bibliography{Schlather}

\end{document}

