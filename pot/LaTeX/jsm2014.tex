\documentclass{beamer}
\usepackage{beamerthemesplit}
% \usepackage{pstricks}
\usepackage{graphicx}
\usepackage{mdwlist}
\usepackage{lineno, hyperref}

\usepackage{amssymb,latexsym,amsmath,amsthm,bbm}
\usepackage{hyperref}
\usepackage{tikz}
\usepackage[english]{babel}
\usepackage[latin1]{inputenc}
\usepackage{multirow}
\usepackage{verbatim}
\usepackage{alltt}
\usepackage{mycommands}

\usepackage{cmbright}
\renewcommand*\familydefault{\sfdefault}
\usepackage[T1]{fontenc}


\definecolor{wp-red}{RGB}{204,0,0}
\definecolor{wp-gray}{RGB}{51,51,51}
\definecolor{reynolds-red}{RGB}{153,0,0}
\definecolor{pyroman-flame}{RGB}{209,81,34}
\definecolor{hunt-yellow}{RGB}{253,215,38}
\definecolor{genomic-green}{RGB}{125,140,31}
\definecolor{innovation-blue}{RGB}{66,126,147}
\definecolor{bio-indigo}{RGB}{65,86,161}

\setbeamercolor{structure}{fg=wp-red}
\setbeamercolor{title}{bg=white, fg=wp-red}  % changes color on title page
\setbeamerfont{title}{series=\bfseries, size=\huge}
\setbeamerfont{author}{series=\bfseries, size=\Large}
\setbeamerfont{institute}{series=\mdseries, size=\large}

\setbeamercolor{frametitle}{bg=wp-red, fg=white}  % changes color at top of frame
\setbeamerfont{frametitle}{series=\bfseries}
\setbeamercolor{title in head/foot}{fg=white, bg=wp-red}  % changes color for title in footer
\setbeamerfont{title in head/foot}{series=\bfseries}
\setbeamercolor{author in head/foot}{fg=white,bg=wp-gray}  % changes color for author in footer
\setbeamerfont{author in head/foot}{series=\bfseries}


\title[Spatiotemporal Modeling of Extreme Events] % (optional, use only with long paper titles)
{
  Spatiotemporal Modeling of Extreme Events
}
\author[S. Morris and B. Reich]{Samuel Morris and Brian Reich}
\institute[NCSU]{North Carolina State University}
\date{}

\begin{document}

\begin{frame}\frametitle{\ }
\begin{center}
	\maketitle
\end{center}
\end{frame}

\begin{frame}{Motivation}
  \begin{itemize} \setlength{\itemsep}{1em}
    \item Average behavior is important to understand, but it does not paint the whole picture.
    \begin{itemize}
      \item e.g. When constructing river levees, engineers need to be able to estimate a 100-year or 1000-year flood levels.
    \end{itemize}
    \item In geostatistical analysis, kriging uses spatial correlation to help inform prediction at unknown locations.
    \item Want to explore computationally easy methods that are available in higher dimensions
  \end{itemize}
\end{frame}

% \begin{frame}{Introduction to extremes}
%   \begin{itemize} \setlength{\itemsep}{0.5em}
%     \item Max-stable processes (Cooley et al., 2012):
%     \begin{itemize}
%       \item Consider a spatial process $x_t(\bs)$, $t = 1, \ldots, T$.
%       \item Let $M_T(\bs) = \left\{ \bigvee_{t=1}^T x_t(\bs_1), \ldots, \bigvee_{t=1}^T x_t(\bs_n) \right\}$
%       \item If there exists normalizing sequences $a_T(\bs)$ and $b_T(\bs)$ such
%       that for all sites, $\bs_i, i = 1, \ldots, d$,
%       \begin{align*}
%         a_T^{-1}(\bs) \left\{ M_T(\bs) - b_T(\bs) \right\} \converged Y(\bs)
%       \end{align*}
%       which has a non-degenerate distribution, then $Y(\bs)$ is a max-stable process.
%     \end{itemize}
%   \end{itemize}
% \end{frame}

\begin{frame}{Standard non-spatial analysis}
  \begin{itemize} \setlength{\itemsep}{0.5em}
    \item Block maxima:
    \begin{itemize}
      \item Uses yearly maxima
      \item Discards many observations
      \item Models are fit using the generalized extreme value distribution
    \end{itemize}
    \item Generalized extreme value distribution (GEV):
    \begin{align*}
      \Pr(Y_j < y) = G_j(y) = \exp \left\{ -\left[ \left(1 + \xi_j \frac{y - \mu_j}{\sigma_j}\right)_+^{-1/\xi_j} \right] \right\}
    \end{align*}
  \end{itemize}
\end{frame}

\begin{frame}{Standard non-spatial analysis}
  \begin{itemize}
    \item Peaks-over-threshold:
    \begin{itemize}
      \item Incorporates more data than block maxima
      \item Select a threshold, $T$, and fit data above the threshold using the generalized Pareto distribution
      \item Autocorrelation may be an issue between observations (e.g. flood levels don't dissipate overnight)
    \end{itemize}
    \item Generalized Pareto distribution (GPD):
    \begin{align*}
      \Pr(Y_j > y | Y_j > T) = F_j(y) = \left( 1 + \xi_j \frac{y - T}{\sigma_j} \right)_+^{-1/\xi_j}
    \end{align*}
  \end{itemize}
\end{frame}

% \begin{frame}{Multivariate representations}
%   \begin{itemize}
%     \item Multivariate distributions:
%     \begin{itemize}
%       \item Assume common standardized max-stable marginal, like unit-Fr\'{e}chet
%       \begin{align*}
%         \Pr(Z < z) = exp(-z^{-1})
%       \end{align*}
%       \item The multivariate representation for the GEV is
%       \begin{align*}
%         \Pr(\bZ \le \bz)  &= G^*(\bz) = \exp(-V(\bz))\\
%                 V(\bs)    &= d \int_{\Delta_d} \bigvee_{i = 1}^d \frac{w_i}{z_i} H(\ddd w)
%       \end{align*}
%       where
%       \begin{itemize}
%         \item $\Delta_d = \{ \bw \in \calR^d_+ \mid w_1 + \cdots + w_d = 1\}$
%         \item $H$ is a probability measure on $\Delta_d$
%         \item $\int_{\Delta_d}w_i H(\ddd w) = 1 / d$ for $i = 1, \ldots, d$.
%       \end{itemize}
%     \end{itemize}
%   \end{itemize}
% \end{frame}

\begin{frame}{Multivariate analysis}
  \begin{itemize} \setlength{\itemsep}{0.5em}
    \item Multivariate max-stable and GPD models have nice features, but they are
    \begin{itemize}
      \item computationally hard to work with
      \item joint distribution only available in low dimension
    \end{itemize}
    \item Pairwise likelihood approach (Huser and Davison, 2014)
  \end{itemize}
\end{frame}

\begin{frame}{Model objectives}
  \begin{itemize} \setlength{\itemsep}{0.5em}
    \item Our objective is to build a model that
    \begin{itemize}
      \item has a flexible tail
      \item has asymptotic spatial dependence
      \item computation on the order of Gaussian models for large space-time datasets
    \end{itemize}
  \end{itemize}
\end{frame}

\begin{frame}{Thresholding data}
  \begin{itemize} \setlength{\itemsep}{0.5em}
    \item We threshold the observed data at a high threshold $T$.
    \item Thresholded data:
    \begin{align*}
      Y_t^*(\bs) = \left\{ \begin{array}{ll}
          Y_t(\bs) \quad & Y_t(\bs) > T\\
          T \quad & Y_t(\bs) \le T
      \end{array}\right.
    \end{align*}
    \item Allows tails of the distribution to speak for themselves.
  \end{itemize}
\end{frame}

\begin{frame}{Spatial skew-$t$ distribution}
  \begin{itemize} \setlength{\itemsep}{0.5em}
    \item Assume observed data $Y_t(\bs)$ come from a skew-$t$ (Zhang and El-Shaarawi, 2012)
    \begin{align*}
      Y_t(\bs) = X_t(\bs)\beta + \alpha z_t + v_t(\bs)
    \end{align*}
    where
    \begin{itemize} \setlength{\itemsep}{0.25em}
      \item $\alpha \in \calR$ controls the skewness
      \item $z_t \iid N_{(0, \infty)}(0, \sigma^2_t)$ is a random effect
      \item $v_t(\bs)$ is a Gaussian process with variance $\sigma^2_t$ and \Matern correlation
      \item $\sigma_t^2 \iid \text{IG}(a, b)$
    \end{itemize}
  \end{itemize}
\end{frame}

\begin{frame}{Spatial skew-$t$ distribution}
  \begin{itemize} \setlength{\itemsep}{0.5em}
    \item \alert{Conditioned} on $z_t$ and $\sigma^2_t$, $Y_t(\bs)$ is Gaussian
    \item Can use standard geostatistical methods to fit this model.
    \item Predictions can be made through kriging.
    \item \alert{Marginalizing} over $z_t$ and $\sigma^2_t$ (via MCMC),
    \begin{align*}
      Y_t(\bs) \sim \text{skew-t}(\mu, \Sigma^*, \alpha, \text{df}=2a)
    \end{align*}
    where
    \begin{itemize}
    	\item $\mu$ is the location
	\item $a$, $b$ are the IG parameters for $\sigma^2_t$
	\item $\Sigma^* = \frac{ b }{ a } \Sigma$ is a scale matrix, and $\Sigma$ is a \Matern covariance matrix
	\item $\alpha \in \calR$ controls the skewness
    \end{itemize}
  \end{itemize}
\end{frame}

\begin{frame}{Long-range dependence}
  \begin{itemize} \setlength{\itemsep}{0.5em}
    \item The $\chi$ coefficient is a measure of extremal spatial correlation
    \begin{align*}
      \chi(\bh) = \Pr(Y_t(\bs) > c \mid Y_t(\bs + \bh) > c)
    \end{align*}
    \item This value shows asymptotic dependence that does not approach 0 as $\bh \rightarrow \infty$ (Padoan, 2011)
    \item Deal with this through a daily random partition.
  \end{itemize}
\end{frame}

\begin{frame}{Simulated $\chi$ plots}
  \centering
  \includegraphics[width=1\linewidth]{./plots/chi-plots.pdf}\\[-0.25in]
  \includegraphics[width=0.2\linewidth]{./plots/chi-legend.pdf}
\end{frame}

\begin{frame}{Random daily partition}
  \begin{itemize} \setlength{\itemsep}{0.5em}
    \item Daily random partition allows $z_t$ and $\sigma^2_t$ to vary by site.
    \begin{align*}
      Y_t(\bs) = X_t(\bs) \beta + \alpha z_t(\bs) + \sigma(\bs) v_t(\bs)
    \end{align*}
    \item Consider a set of daily knots $\{w_{t1}, \ldots, w_{tK}\}$ that define a daily partition
    $P_{t1}, \ldots, P_{tK}$ such that
    \begin{align*}
      P_{tk} = \{s : k = \argmin_\ell|| \bs - w_{t\ell}|| \}
    \end{align*}
    \item For $\bs \in P_{tk}$
    \begin{align*}
      z_t(\bs) &= z_{tk}\\
      \sigma^2_t(\bs) &= \sigma^2_{tk}
    \end{align*}
    \item Within each partition $Y_t(\bs)$ has the same MVT distribution as before.
  \end{itemize}
\end{frame}

\begin{frame}{Example daily partition}
	Two sample partitions
    \centering
    \includegraphics[width=0.54\linewidth]{./plots/example-partition-1.pdf}
    \includegraphics[width=0.54\linewidth]{./plots/example-partition-2.pdf}
\end{frame}

\begin{frame}{MCMC details}
  \begin{itemize} \setlength{\itemsep}{0.5em}
    \item Three main steps:
    \begin{enumerate}[1.]
      \item Impute missing observations and data below $T$
      \item Update parameters with standard random walk Metropolis Hastings or Gibbs sampling
      \item Make spatial predictions
    \end{enumerate}
    \item Priors are selected to be conjugate when possible.
  \end{itemize}
\end{frame}

\begin{frame}{Data analysis}
  \begin{itemize} \setlength{\itemsep}{0.5em}
    \item Ozone analysis at 85 sites in NC, SC, and GA for 92 days
    \includegraphics[width=1\linewidth]{./plots/ozone_station.pdf}
  \end{itemize}
\end{frame}

\begin{frame}{Model comparisons}
  \begin{itemize} \setlength{\itemsep}{0.5em}
    \item 9 different analysis methods incorporating
    \begin{itemize}
      \item Gaussian vs $t$ vs skew-$t$ marginal distribution
      \item $K=1$ partition vs $K=5$ partitions
      \item No thresholding vs thresholded
      \begin{itemize}
        \item Thresholded data at $T=0.90$ sample quantile
      \end{itemize}
    \end{itemize}
    \item All methods use a \Matern or exponential covariance ($\nu = 0.5$)
    \item Compare quantile and Brier scores using 5-fold cross validation (Gneiting and Raftery, 2007)
    \item Mean function modeled using a first-order spatial trend
  \end{itemize}
\end{frame}

\begin{frame}{Quantile score}
  \begin{itemize} \setlength{\itemsep}{0.5em}
    \item The quantile score for the $\tau$th quantile is
    \begin{align*}
      2 \{ I[y < \widehat{q}(\tau)] - \tau\} (\widehat{q} - y)
    \end{align*}
    where:
    \begin{itemize}
      \item $y$ is a test set value
      \item $\widehat{q}(\tau)$ is the estimated $\tau$th quantile
    \end{itemize}
  \end{itemize}
\end{frame}

\begin{frame}{Brier score}
  \begin{itemize} \setlength{\itemsep}{0.5em}
	\item Brier score for predicting exceedance of threshold $c$
	\begin{align*}
	  [e(c) - P(c)]^2
	\end{align*}
	where 
	\begin{itemize}
		\item $y$ is a test set value
		\item $e(c) = I[y > c]$
		\item $P(c)$ is the predicted probability of exceeding $c$
	\end{itemize}
  \end{itemize}
\end{frame}

\begin{frame}{Five-fold cross-validation results}
  \begin{table}[htbp]
    \small
    \centering
    \begin{tabular}{l|c|r|rrrrr}
          \multicolumn{3}{c}{\ } & \multicolumn{5}{c}{Quantile}\\
           \hline
  Marginal & $K$ & $T$  & 0.900 & 0.950 & 0.990 & 0.995 & 0.999\\
  \hline
Gaussian & 1 & 0 & 16.38 & 15.76 & 14.52 & 14.08 & 13.22\\
$t$ & 1 & 0 & 16.15 & 15.51 & 14.00 & 13.43 & 12.32\\
$t$ & 5 & 0 & 13.61 & 12.66 & 10.96 & 10.40 & 9.34\\
skew $t$ & 1 & 0 & 9.24  & 7.27 & 4.13  & 3.27  & 1.96\\
skew $t$ & 5 & 0 & 15.81 & 14.46 & 11.57 & 10.57 & 8.60\\
$t$ & 1 & 0.9 & 5.52  & 3.58  & 1.77  & 1.47  & 1.10\\
$t$ & 5 & 0.9 & 5.98  & 4.27  & 2.41  & 2.03  & 1.49\\
 skew $t$ & 1 & 0.9 & {\bf 4.91}  & {\bf 3.16} & {\bf 1.45}  & {\bf 1.16}  & {\bf 0.82}\\
skew $t$ & 3 & 0.9 & 5.58 & 3.78 & 1.93& 1.58& 1.11\\
\hline
    \end{tabular}
  \end{table}
  \begin{itemize}
  	\item Brier score results are similar.
  \end{itemize}
\end{frame}

\begin{frame}{Simulation study settings}
  \begin{itemize} \setlength{\itemsep}{0.5em}
    \item Data generated from 6 different settings.
    \begin{enumerate}[1.]
      \item Gaussian
      \item $t$-1 with 4 degrees of freedom
      \item $t$-5 with 4 degrees of freedom
      \item skew $t$-1 with 4 degrees of freedom ($\alpha = 3$)
      \item skew $t$-5 with 4 degrees of freedom ($\alpha = 3$)
      \item Half-Gaussian, Half $t$-1 (spatial range = 0.4)
    \end{enumerate}
    \item Spatial settings
    \begin{itemize}
      \item $\bs \in [0, 1] \times [0, 1]$
      \item Exponential covariance with range: 0.1
      \item Ratio of spatial to nugget error: 0.9
    \end{itemize}
  \end{itemize}
\end{frame}

\begin{frame}{Simulation study methods}
  \begin{itemize} \setlength{\itemsep}{0.5em}
    \item 5 different analysis methods
    \begin{enumerate}[1.]
      \item Gaussian
      \item Skew $t$-1 ($T=0$)
      \item Skew $t$-1 ($T=0.9$)
      \item Skew $t$-5 ($T=0$)
      \item Skew $t$-5 ($T=0.9$)
    \end{enumerate}
    \item All methods use a \Matern covariance structure except for method 5 which uses an exponential covariance ($\nu = 0.5$)
  \end{itemize}
\end{frame}

\begin{frame}{Simulation study results}
  \begin{itemize} \setlength{\itemsep}{0.5em}
    \item Results are similar to the results from the data analysis
    \item Biggest gains come from thresholding.
    \item Using skew models give additional gain, but small relative to gain for thresholding.
  \end{itemize}
\end{frame}

\begin{frame}{Future work}
  \begin{itemize} \setlength{\itemsep}{0.5em}
    \item Comparison with extreme value analysis methods
    \item Including time in the model
    \begin{itemize}
      \item AR(1): $Y_t(\bs) = X_t(\bs) \beta + \phi Y_{t-1}(\bs) + \alpha z_t(\bs) + v_t(\bs)$
    \end{itemize}
  \end{itemize}
\end{frame}

\begin{frame}{Questions}
  \begin{itemize} \setlength{\itemsep}{0.5em}
    \item Any questions?
    \item Thank you for your attention.
  \end{itemize}
\end{frame}

\begin{frame}{References}
  \begin{itemize} \setlength{\itemsep}{0.5em}
    \item Huser, R. and Davison, A. C. (2014) Space-time modelling of extreme events. {\it Journal of the Royal Statistical Society: Series B (Statistical Methodology)}, {\bf 76}, 439--461.
    \item Padoan, S. A. (2011) Multivariate extreme models based on underlying skew-$t$ and skew-normal distributions. {\it Journal of Multivariate Analysis}, {\bf 102}, 977--991.
    \item Zhang, H. and El-Shaarawi, A. (2010) On spatial skew-Gaussian processes and applications. {\it Environmetrics}, {\bf 21}, 33--47.
  \end{itemize}
\end{frame}

\end{document}