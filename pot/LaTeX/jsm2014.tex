\documentclass[t]{beamer}
\usepackage{beamerthemesplit}
\usepackage{pstricks}
\usepackage{graphicx}
\usepackage{mdwlist}

\usepackage{amssymb,latexsym,amsmath,amsthm,bbm}
\usepackage{tikz}
\usepackage[english]{babel}
\usepackage[latin1]{inputenc}
\usepackage{multirow}
\usepackage{verbatim}
\usepackage{alltt}
\usepackage{mycommands}

\usepackage{cmbright}
\renewcommand*\familydefault{\sfdefault}
\usepackage[T1]{fontenc}


\definecolor{wp-red}{RGB}{204,0,0}
\definecolor{wp-gray}{RGB}{51,51,51}
\definecolor{reynolds-red}{RGB}{153,0,0}
\definecolor{pyroman-flame}{RGB}{209,81,34}
\definecolor{hunt-yellow}{RGB}{253,215,38}
\definecolor{genomic-green}{RGB}{125,140,31}
\definecolor{innovation-blue}{RGB}{66,126,147}
\definecolor{bio-indigo}{RGB}{65,86,161}

\setbeamercolor{structure}{fg=wp-red}
\setbeamercolor{title}{bg=white, fg=wp-red}  % changes color on title page
\setbeamerfont{title}{series=\bfseries, size=\LARGE}
\setbeamerfont{author}{series=\bfseries}

\setbeamercolor{frametitle}{bg=wp-red, fg=white}  % changes color at top of frame
\setbeamerfont{frametitle}{series=\mdseries}
\setbeamercolor{title in head/foot}{fg=white, bg=wp-red}  % changes color for title in footer
\setbeamerfont{title in head/foot}{series=\mdseries}
\setbeamercolor{author in head/foot}{fg=white,bg=wp-gray}  % changes color for author in footer
\setbeamerfont{author in head/foot}{series=\mdseries}


\title[Spatiotemporal Modeling of Extreme Events] % (optional, use only with long paper titles)
{
  Spatiotemporal Modeling of Extreme Events
}
\author[B. Reich and S. Morris]{Brian Reich and Samuel Morris}
\institute[NCSU]{North Carolina State University}
\date{August 7, 2014}

\begin{document}

\begin{frame}
\begin{center}
  \titlepage
\end{center}
\end{frame}

\begin{frame}{Motivation}
  \begin{itemize}
    \item Average behavior is important to understand, but it does not paint the whole picture.
    \begin{itemize}
      \item e.g. When constructing river levees, engineers need to be able to estimate a 100-year or 1000-year flood levels.
    \end{itemize}
    \item In geostatistical analysis, kriging uses spatial correlation to help inform prediction at unknown locations.
    \item Want to explore ways to incorporate this spatial correlation when estimating the tails of the distribution.
  \end{itemize}
\end{frame}

\begin{frame}{Introduction to extremes}
  \begin{itemize} \itemsep 1.5em
  \item Two common methods for extremes:
  \begin{enumerate}[1]
    \item Block maxima
    \begin{itemize}
      \item Only use maxima of independent blocks (e.g. yearly maxima)
      \item Lose information by throwing away non-maxima
    \end{itemize}
    \item Peaks over threshold
    \begin{itemize}
      \item Use all observations over a threshold
      \item
    \end{itemize}
  \end{enumerate}

  \end{itemize}
\end{frame}



\end{document}