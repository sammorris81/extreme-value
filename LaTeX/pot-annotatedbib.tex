\documentclass[11pt]{article}
\usepackage{amssymb, amsthm, amsmath}
\usepackage{bm}
\usepackage{graphicx}
\usepackage[authoryear]{natbib}
\usepackage{bm}
\usepackage{verbatim}
\usepackage{lineno}
\usepackage{times}
\usepackage{soul}
\usepackage{color}

\usepackage[left=1in,top=1in,right=1in]{geometry}
\newcommand{\btheta}{ \mbox{\boldmath $\theta$}}
\newcommand{\bmu}{ \mbox{\boldmath $\mu$}}
\newcommand{\balpha}{ \mbox{\boldmath $\alpha$}}
\newcommand{\bbeta}{ \mbox{\boldmath $\beta$}}
\newcommand{\bdelta}{ \mbox{\boldmath $\delta$}}
\newcommand{\blambda}{ \mbox{\boldmath $\lambda$}}
\newcommand{\bgamma}{ \mbox{\boldmath $\gamma$}}
\newcommand{\brho}{ \mbox{\boldmath $\rho$}}
\newcommand{\bpsi}{ \mbox{\boldmath $\psi$}}
\newcommand{\bepsilon}{ \mbox{\boldmath $\epsilon$}}
\newcommand{\bomega}{ \mbox{\boldmath $\omega$}}
\newcommand{\bOmega}{ \mbox{\boldmath $\Omega$}}
\newcommand{\bDelta}{ \mbox{\boldmath $\Delta$}}
\newcommand{\bSigma}{ \mbox{\boldmath $\Sigma$}}
\newcommand{\bPsi}{\mbox{\boldmath $\Psi$}}
\newcommand{\bOne}{\mbox{\boldmath $1$}}
\newcommand{\omu}{\overline{\mu}}
\newcommand{\oSigma}{\overline{\Sigma}}
\newcommand{\Yt}{{\tilde Y}}
\newcommand{\bA}{ \mbox{\bf A}}
\newcommand{\bP}{ \mbox{\bf P}}
\newcommand{\bx}{ \mbox{\bf x}}
\newcommand{\bX}{ \mbox{\bf X}}
\newcommand{\bB}{ \mbox{\bf B}}
\newcommand{\bZ}{ \mbox{\bf Z}}
\newcommand{\by}{ \mbox{\bf y}}
\newcommand{\bY}{ \mbox{\bf Y}}
\newcommand{\bz}{ \mbox{\bf z}}
\newcommand{\bh}{ \mbox{\bf h}}
\newcommand{\br}{ \mbox{\bf r}}
\newcommand{\bt}{ \mbox{\bf t}}
\newcommand{\bs}{ \mbox{\bf s}}
\newcommand{\bb}{ \mbox{\bf b}}
\newcommand{\bL}{ \mbox{\bf L}}
\newcommand{\bu}{ \mbox{\bf u}}
\newcommand{\bv}{ \mbox{\bf v}}
\newcommand{\bV}{ \mbox{\bf V}}
\newcommand{\bW}{ \mbox{\bf W}}
\newcommand{\bG}{ \mbox{\bf G}}
\newcommand{\bH}{ \mbox{\bf H}}
\newcommand{\bw}{ \mbox{\bf w}}
\newcommand{\bo}{ \mbox{\bf o}}
\newcommand{\bfe}{ \mbox{\bf e}}
\newcommand{\iid}{\stackrel{iid}{\sim}}
\newcommand{\indep}{\stackrel{indep}{\sim}}
\newcommand{\calR}{{\cal R}}
\newcommand{\calG}{{\cal G}}
\newcommand{\calD}{{\cal D}}
\newcommand{\calS}{{\cal S}}
\newcommand{\calB}{{\cal B}}
\newcommand{\calA}{{\cal A}}
\newcommand{\calT}{{\cal T}}
\newcommand{\calO}{{\cal O}}
\newcommand{\argmax}{{\mathop{\rm arg\, max}}}
\newcommand{\argmin}{{\mathop{\rm arg\, min}}}
\newcommand{\Frechet}{\mbox{Fr$\acute{\mbox{e}}$chet }}
\newcommand{\Matern}{\mbox{Mat$\acute{\mbox{e}}$rn }}
\newcommand{\ballunion}{B_a(\bs_1) \cup B_b(\bs_2) }

\newcommand{\beq}{ \begin{equation}}
\newcommand{\eeq}{ \end{equation}}
\newcommand{\beqn}{ \begin{eqnarray}}
\newcommand{\eeqn}{ \end{eqnarray}}

\pdfpageheight 11in
\pdfpagewidth 8.5in
\linespread{1.0}

\begin{document}\linenumbers
\section*{\citep{Azzalini1996}}
Consider $\bY \in \calR^k$ that is distributed as MVN and is independent from $Y_0 \sim N(0, 1)$. Then for
\begin{align}
\left( \begin{array}{c}
    Y_0\\
    \bY
\end{array} \right)
\sim
N_{k + 1} \left\{ 0, \left( 
    \begin{array}{c c}
    1 & 0 \\
    0 & \bPsi
    \end{array}
\right) \right\}
\end{align}
Then, $Z_j = \delta_j |Y_0| + \sqrt{ 1 - \delta_j^2} Y_j, j = 1, \ldots, k$ is skewed normal and its density is
\begin{align}
    f_k(z) = 2 \phi_k (z; \bOmega) \Phi (\balpha^t \bz) \quad \bz \in \calR^k
\end{align}

\section*{\citep{Azzalini1999}}

\section*{\citep{Branco2001}}
Modeling distributions that can account for skewness and heavy tails.

\subsection*{Multivariate elliptical}
Notation: $\bX \sim El_k(\bmu, \bSigma; \phi)$ means that $\bX \in \calR^k$ follows an elliptical distribution with location vector $\bmu \in \calR^k$, a dispersion matrix $\bSigma \in \calR^{k \times k}$ and characteristic function $\phi$.
If the density exists, then it is given by
\beqn
    f(\bx \mid \bmu, \bSigma) = | \bSigma |^{-1/2} g^{(k)} [ ( \bx - \bmu)^T \bSigma^{-1} (\bx - \bmu)]
\eeqn

\subsection*{Multivariate skew elliptical}
Notation: $\bY \sim SE_k(\bmu, \bOmega, \bdelta; \phi)$ means that $\bY \in \calR^k$ follows a skew-elliptical distribution with location vector $\bmu \in \calR^k$, a dispersion matrix $\bSigma \in \calR^{k \times k}$, characteristic function $\phi$ and skewness parameter $\bdelta$.
If the density exists, then it is given by
\beqn
    f_{\bY}(\by) = 2 f_{g^(k)} (\by) F_{g_{q(\by)}} (\blambda^T (\by - \bmu))
\eeqn

\subsection*{Multivariate skew normal distribution}
For a multivariate skew normal distribution, the density function is
\beqn
    f_{\bY}(\by) = 2 \phi_k ( \by; \bmu, \bOmega) \Phi(\blambda^T (\by - \bmu))
\eeqn
where
\beqn
    \blambda^T = \frac{ \bdelta^T \bOmega^{-1}}{(1 - \bdelta^T \bOmega^{-1} \bdelta)^{1/2}}
\eeqn

\subsection*{Multivariate skew $t$ distribution}
For a multivariate skew $t$ distribution, the density function is 
\beqn
    f_{\bY}(\by) = 2f_{\nu, \tau}(\by; \bmu; \bOmega) F_{\nu^*, \tau^*}(\blambda^T(\by - \bmu))
\eeqn

\subsection*{Other densities mentioned}
\begin{itemize}
    \item Skew logistic
    \item Skew stable distribution
    \item Skew exponential power distribution
    \item Skew Pearson Type II distribution
\end{itemize}

\section*{\citep{Sahu2003}}

\section*{\citep{Gupta2004}}
p. 189
The general multivariate skew normal distribution has density function
\beqn
    2 \phi_k (z; \Omega) \Phi (\alpha^{\text{T}} z) \qquad (z \in \calR^k)
\eeqn
where $\phi_k(z; \Omega)$ is a $k$-dimensional process with mean zero, and correlation matrix $\Omega$, $\Phi( \cdot)$ is the $N(0, 1)$ distribution, and $\alpha \in \calR^k$ is a shape term.

\section*{\citep{Allard2007}}

\section*{\citep{Zhang2010}}

\section*{\citep{Minozzo2012}}
p. 164
Let $U_t \sim N(0, 1)$ and $V_t(\bs) \sim MVN$ with mean 0, and variance 1, then 
\beqn
    \bY_t(\bs) = \sigma \delta |U_t| + \sigma \sqrt{1 - \delta^2} V_t(\bs)
\eeqn
follows a 

\bibliographystyle{rss}
\bibliography{Schlather}
\end{document}