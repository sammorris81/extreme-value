\documentclass[11pt]{article}
\usepackage[letterpaper, margin=1in]{geometry}
% \documentclass[useAMS,usenatbib,referee]{biomweb}
\usepackage{amssymb, amsthm, amsmath}
\usepackage{amssymb, amsmath}
\usepackage{bm}
\usepackage{graphicx}
\usepackage[authoryear]{natbib}
\usepackage{bm}
\usepackage{verbatim}
\usepackage{lineno}
\usepackage{soul}
\usepackage{color}
% \usepackage{enumitem}
\usepackage{enumerate}
\usepackage{setspace}
\usepackage{appendix}
% \usepackage{caption}
% \captionsetup[table]{name=Web Table}

% cross-referencing appendices
\usepackage{xr}
\externaldocument{skewt_rev}
\externaldocument{supplemental_rev}

% \usepackage[left=1in,top=1in,right=1in]{geometry}
% \pdfpageheight 11in
% \pdfpagewidth 8.5in
% \linespread{2.0}
\newcommand{\btheta}{ \mbox{\boldmath $\theta$}}
\newcommand{\bmu}{ \mbox{\boldmath $\mu$}}
\newcommand{\balpha}{ \mbox{\boldmath $\alpha$}}
\newcommand{\bbeta}{ \mbox{\boldmath $\beta$}}
\newcommand{\bdelta}{ \mbox{\boldmath $\delta$}}
\newcommand{\blambda}{ \mbox{\boldmath $\lambda$}}
\newcommand{\bgamma}{ \mbox{\boldmath $\gamma$}}
\newcommand{\brho}{ \mbox{\boldmath $\rho$}}
\newcommand{\bpsi}{ \mbox{\boldmath $\psi$}}
\newcommand{\bepsilon}{ \mbox{\boldmath $\epsilon$}}
\newcommand{\bomega}{ \mbox{\boldmath $\omega$}}
\newcommand{\bOmega}{ \mbox{\boldmath $\Omega$}}
\newcommand{\bDelta}{ \mbox{\boldmath $\Delta$}}
\newcommand{\bSigma}{ \mbox{\boldmath $\Sigma$}}
\newcommand{\bPsi}{\mbox{\boldmath $\Psi$}}
\newcommand{\bOne}{\mbox{\boldmath $1$}}
\newcommand{\omu}{\overline{\mu}}
\newcommand{\oSigma}{\overline{\Sigma}}
\newcommand{\Yt}{{\tilde Y}}
\newcommand{\bA}{ \mbox{\bf A}}
\newcommand{\bP}{ \mbox{\bf P}}
\newcommand{\bx}{ \mbox{\bf x}}
\newcommand{\bX}{ \mbox{\bf X}}
\newcommand{\bB}{ \mbox{\bf B}}
\newcommand{\bZ}{ \mbox{\bf Z}}
\newcommand{\by}{ \mbox{\bf y}}
\newcommand{\bY}{ \mbox{\bf Y}}
\newcommand{\bz}{ \mbox{\bf z}}
\newcommand{\bh}{ \mbox{\bf h}}
\newcommand{\br}{ \mbox{\bf r}}
\newcommand{\bt}{ \mbox{\bf t}}
\newcommand{\bs}{ \mbox{\bf s}}
\newcommand{\bb}{ \mbox{\bf b}}
\newcommand{\bL}{ \mbox{\bf L}}
\newcommand{\bu}{ \mbox{\bf u}}
\newcommand{\bv}{ \mbox{\bf v}}
\newcommand{\bV}{ \mbox{\bf V}}
\newcommand{\bW}{ \mbox{\bf W}}
\newcommand{\bG}{ \mbox{\bf G}}
\newcommand{\bH}{ \mbox{\bf H}}
\newcommand{\bw}{ \mbox{\bf w}}
\newcommand{\bo}{ \mbox{\bf o}}
\newcommand{\bfe}{ \mbox{\bf e}}
\newcommand{\iid}{\stackrel{iid}{\sim}}
\newcommand{\indep}{\stackrel{indep}{\sim}}
\newcommand{\calR}{{\cal R}}
\newcommand{\calG}{{\cal G}}
\newcommand{\calD}{{\cal D}}
\newcommand{\calS}{{\cal S}}
\newcommand{\calB}{{\cal B}}
\newcommand{\calA}{{\cal A}}
\newcommand{\calT}{{\cal T}}
\newcommand{\calO}{{\cal O}}
\newcommand{\argmax}{{\mathop{\rm arg\, max}}}
\newcommand{\argmin}{{\mathop{\rm arg\, min}}}
\newcommand{\Frechet}{\mbox{Fr$\acute{\mbox{e}}$chet }}
\newcommand{\Matern}{\mbox{Mat$\acute{\mbox{e}}$rn }}
\newcommand{\ballunion}{B_a(\bs_1) \cup B_b(\bs_2) }

\newcommand{\beq}{ \begin{equation}}
\newcommand{\eeq}{ \end{equation}}
\newcommand{\beqn}{ \begin{eqnarray}}
\newcommand{\eeqn}{ \end{eqnarray}}
\begin{document}

\begin{center}
Point-by-Point response to Report on revised manuscript\\
Biometrics MS \#150926M\\
\today
\end{center}

Again, we wish to thank the reviewers and editors for their helpful comments and feedback regarding our manuscript. 
To address the concerns, we edited the manuscript to address the concerns given in the revised report.

\section*{Response to the comments of the Associate Editor}
\subsection*{Remaining comments}
\begin{enumerate}[1.]
	\item p.2 1.1-2: ``Asymptotic \ldots level'' this sentence doesn't make sense

	\begin{response}
	We have changed this sentence to read ``Asymptotic dependence/independence are notions which describe the probability that two random variables simultaneously exceed an extremely high level.''
	\end{response}
	
	\item p.6 Section 3.2 1.2: space missing after .
	
	\begin{response}
	We have added a space after the `.'
	\end{response}
	
	\item p.7 Figure 2: is this really an \emph{estimate} of $\chi(h)$; are these not theoretical values?
	
	\begin{response}
	We have changed the wording on p.7 (2 lines above the suggested location of Figure 2) text describing Figure 2 from ``we estimate $\chi(h)$ \ldots'' to ``we show $\chi(h)$ \ldots''.
	\end{response}

	\item The phrasing ``we use a transformation of a Gaussian random variable on $z_t$\ldots'' sounds quite strange. Why not just say ``we transform $z_t$ to the Gaussian scale using the probability integral transform''? And similarly for $\sigma^2_t$.
	
	\begin{response}
	We have adopted the wording suggested by the reviewer.
	\end{response}
	
	\item p.7 Section 4.1: What method do you use to simulate from the truncated multivariate Gaussian - rejection? It would be useful to give an idea of how difficult or otherwise this step is.
	
	\begin{response}
	We have added clarification that we use the conditional normal distribution to sample from the univariate truncated normal distribution.
	\end{response}
	
	\item p.11 Section 5.1 / Fig. 3: the title of the 4th panel in Figure 3 is ``Asymmetric logistic'' whereas (4) in 5.1 is ``Reich and Shaby'' process
	
	\begin{response}
	We have changed the title of the plot to ``Reich and Shaby'' to be consistent with the text.
	\end{response}
	
	\item p.11 1-2. \texttt{rmaxstab}
	
	\begin{response}
	This has been corrected.
	\end{response}
	
	\item p.12 item (5) typo ``max-table''
	
	\begin{response}
	This has been corrected.
	\end{response}
	
	\item p.15 Section 6.1: when giving the burn in period of 25,000, is this taken out of the 30,000 iterations, or in addition to them?
	
	\begin{response}
	We have adjusted the wording (as well as the corresponding part of the simulation study in 5.1) here to reflect that the burn-in period is taken out of the number of iterations reported.
	\end{response}
	
	\item Web Appendix C, p.6 final line: typo "Poission"; p.7 eq.(3) typo $>=$
	
	\begin{response}
	These have been corrected.		
	\end{response}

\end{enumerate}
\bibliographystyle{biom}
\bibliography{library}

\end{document}